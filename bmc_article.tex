%% BioMed_Central_Tex_Template_v1.06
%%                                      %
%  bmc_article.tex            ver: 1.06 %
%                                       %

%%IMPORTANT: do not delete the first line of this template
%%It must be present to enable the BMC Submission system to
%%recognise this template!!

%%%%%%%%%%%%%%%%%%%%%%%%%%%%%%%%%%%%%%%%%
%%                                     %%
%%  LaTeX template for BioMed Central  %%
%%     journal article submissions     %%
%%                                     %%
%%          <8 June 2012>              %%
%%                                     %%
%%                                     %%
%%%%%%%%%%%%%%%%%%%%%%%%%%%%%%%%%%%%%%%%%


%%%%%%%%%%%%%%%%%%%%%%%%%%%%%%%%%%%%%%%%%%%%%%%%%%%%%%%%%%%%%%%%%%%%%
%%                                                                 %%
%% For instructions on how to fill out this Tex template           %%
%% document please refer to Readme.html and the instructions for   %%
%% authors page on the biomed central website                      %%
%% http://www.biomedcentral.com/info/authors/                      %%
%%                                                                 %%
%% Please do not use \input{...} to include other tex files.       %%
%% Submit your LaTeX manuscript as one .tex document.              %%
%%                                                                 %%
%% All additional figures and files should be attached             %%
%% separately and not embedded in the \TeX\ document itself.       %%
%%                                                                 %%
%% BioMed Central currently use the MikTex distribution of         %%
%% TeX for Windows) of TeX and LaTeX.  This is available from      %%
%% http://www.miktex.org                                           %%
%%                                                                 %%
%%%%%%%%%%%%%%%%%%%%%%%%%%%%%%%%%%%%%%%%%%%%%%%%%%%%%%%%%%%%%%%%%%%%%

%%% additional documentclass options:
%  [doublespacing]
%  [linenumbers]   - put the line numbers on margins

%%% loading packages, author definitions

%\documentclass[twocolumn]{bmcart}% uncomment this for twocolumn layout and comment line below
\documentclass{bmcart}

%%% Load packages
%\usepackage{amsthm,amsmath}
%\RequirePackage{natbib}
%\RequirePackage[authoryear]{natbib}% uncomment this for author-year bibliography
%\RequirePackage{hyperref}
\usepackage[utf8]{inputenc} %unicode support
%\usepackage[applemac]{inputenc} %applemac support if unicode package fails
%\usepackage[latin1]{inputenc} %UNIX support if unicode package fails
\usepackage{mathtools}
\usepackage[pdf]{graphviz}


%%%%%%%%%%%%%%%%%%%%%%%%%%%%%%%%%%%%%%%%%%%%%%%%%
%%                                             %%
%%  If you wish to display your graphics for   %%
%%  your own use using includegraphic or       %%
%%  includegraphics, then comment out the      %%
%%  following two lines of code.               %%
%%  NB: These line *must* be included when     %%
%%  submitting to BMC.                         %%
%%  All figure files must be submitted as      %%
%%  separate graphics through the BMC          %%
%%  submission process, not included in the    %%
%%  submitted article.                         %%
%%                                             %%
%%%%%%%%%%%%%%%%%%%%%%%%%%%%%%%%%%%%%%%%%%%%%%%%%


\def\includegraphic{}
\def\includegraphics{}



%%% Put your definitions there:
\startlocaldefs
\endlocaldefs


%%% Begin ...
\begin{document}

%%% Start of article front matter
\begin{frontmatter}

\begin{fmbox}
\dochead{Research}

%%%%%%%%%%%%%%%%%%%%%%%%%%%%%%%%%%%%%%%%%%%%%%
%%                                          %%
%% Enter the title of your article here     %%
%%                                          %%
%%%%%%%%%%%%%%%%%%%%%%%%%%%%%%%%%%%%%%%%%%%%%%

\title{A hybrid P. Falciparum transmission model}

%%%%%%%%%%%%%%%%%%%%%%%%%%%%%%%%%%%%%%%%%%%%%%
%%                                          %%
%% Enter the authors here                   %%
%%                                          %%
%% Specify information, if available,       %%
%% in the form:                             %%
%%   <key>={<id1>,<id2>}                    %%
%%   <key>=                                 %%
%% Comment or delete the keys which are     %%
%% not used. Repeat \author command as much %%
%% as required.                             %%
%%                                          %%
%%%%%%%%%%%%%%%%%%%%%%%%%%%%%%%%%%%%%%%%%%%%%%

\author[
   addressref={aff1},                   % id's of addresses, e.g. {aff1,aff2}
   corref={aff1},                       % id of corresponding address, if any
   noteref={n1},                        % id's of article notes, if any
   email={giovanni.charles10@imperial.ac.uk}   % email address
]{\inits{GC}\fnm{Giovanni} \snm{Charles}}

% TODO: other authors (Ellie, Azra)
%\author[
%   addressref={aff1,aff2},
%   email={john.RS.Smith@cambridge.co.uk}
%]{\inits{PW}\fnm{Peter} \snm{Winskill}}

%%%%%%%%%%%%%%%%%%%%%%%%%%%%%%%%%%%%%%%%%%%%%%
%%                                          %%
%% Enter the authors' addresses here        %%
%%                                          %%
%% Repeat \address commands as much as      %%
%% required.                                %%
%%                                          %%
%%%%%%%%%%%%%%%%%%%%%%%%%%%%%%%%%%%%%%%%%%%%%%

\address[id=aff1]{%                           % unique id
  \orgname{Department of Infectious Disease, Imperial College London}, % university, etc
  \street{Praed Street},                      %
  %\postcode{}                                % post or zip code
  \city{London},                              % city
  \cny{UK}                                    % country
}
%\address[id=aff2]{%
%  \orgname{Marine Ecology Department, Institute of Marine Sciences Kiel},
%  \street{D\"{u}sternbrooker Weg 20},
%  \postcode{24105}
%  \city{Kiel},
%  \cny{Germany}
%}

%%%%%%%%%%%%%%%%%%%%%%%%%%%%%%%%%%%%%%%%%%%%%%
%%                                          %%
%% Enter short notes here                   %%
%%                                          %%
%% Short notes will be after addresses      %%
%% on first page.                           %%
%%                                          %%
%%%%%%%%%%%%%%%%%%%%%%%%%%%%%%%%%%%%%%%%%%%%%%

\begin{artnotes}
%\note{Sample of title note}     % note to the article
\note[id=n1]{Equal contributor} % note, connected to author
\end{artnotes}

\end{fmbox}% comment this for two column layout

%%%%%%%%%%%%%%%%%%%%%%%%%%%%%%%%%%%%%%%%%%%%%%
%%                                          %%
%% The Abstract begins here                 %%
%%                                          %%
%% Please refer to the Instructions for     %%
%% authors on http://www.biomedcentral.com  %%
%% and include the section headings         %%
%% accordingly for your article type.       %%
%%                                          %%
%%%%%%%%%%%%%%%%%%%%%%%%%%%%%%%%%%%%%%%%%%%%%%

\begin{abstractbox}

\begin{abstract} % abstract
\parttitle{Background}
Text for this section.

\parttitle{Methods}
Text for this section.

\parttitle{Results}
Text for this section.

\parttitle{Conclusions}
Text for this section.
\end{abstract}

%%%%%%%%%%%%%%%%%%%%%%%%%%%%%%%%%%%%%%%%%%%%%%
%%                                          %%
%% The keywords begin here                  %%
%%                                          %%
%% Put each keyword in separate \kwd{}.     %%
%%                                          %%
%%%%%%%%%%%%%%%%%%%%%%%%%%%%%%%%%%%%%%%%%%%%%%

\begin{keyword}
\kwd{sample}
\kwd{article}
\kwd{author}
\end{keyword}

% MSC classifications codes, if any
%\begin{keyword}[class=AMS]
%\kwd[Primary ]{}
%\kwd{}
%\kwd[; secondary ]{}
%\end{keyword}

\end{abstractbox}
%
%\end{fmbox}% uncomment this for twcolumn layout

\end{frontmatter}

%%%%%%%%%%%%%%%%%%%%%%%%%%%%%%%%%%%%%%%%%%%%%%
%%                                          %%
%% The Main Body begins here                %%
%%                                          %%
%% Please refer to the instructions for     %%
%% authors on:                              %%
%% http://www.biomedcentral.com/info/authors%%
%% and include the section headings         %%
%% accordingly for your article type.       %%
%%                                          %%
%% See the Results and Discussion section   %%
%% for details on how to create sub-sections%%
%%                                          %%
%% use \cite{...} to cite references        %%
%%  \cite{koon} and                         %%
%%  \cite{oreg,khar,zvai,xjon,schn,pond}    %%
%%  \nocite{smith,marg,hunn,advi,koha,mouse}%%
%%                                          %%
%%%%%%%%%%%%%%%%%%%%%%%%%%%%%%%%%%%%%%%%%%%%%%

%%%%%%%%%%%%%%%%%%%%%%%%% start of article main body
% <put your article body there>

%%%%%%%%%%%%%%%%
%% Background %%
%%
\section*{Background}
We present a model for exploring combined Malaria interventions. The individual-based human and adult mosquito models allow for fine-grained heterogeneity, stochastic outputs, detailed intervention strategies. This is combined with a compartmental model for aquatic (and optionally adult) mosquitoes for fast modelling of emergence dynamics.

\section*{Methods}

We present a formulation of P.Falciparum transmission in the presence of interventions. We also include implementation details for the individual based models.

\subsection*{Vector biology}

The vector model was split between the aquatic and adult-female stage

\subsubsection*{Aquatic model}

This compartmental model tracks state of mosquitoes in larval and pupal states. This dictates the emergence of adult female mosquitoes into latter parts of the model. This is from \cite{griffin_reducing_2010}.

\begin{gather*}
    \frac{dE^v}{dt} = \beta_{egg}^v(t) M - \frac{E^v}{de} - \mu_e E^v \left(1 + \frac{E^v + L^v}{K}\right) \\
    \frac{dL^v}{dt} = \frac{E^v}{de} - \frac{L^v}{dl} - \mu_l L^v \left(1 + \gamma \frac{E^v + L^v}{K}\right)  \\
    \frac{dP^v}{dt} = \frac{L^v}{dl} - \frac{P^v}{dp} - \mu_p P^v
\end{gather*}

Carrying capacity drives the mortality of early and late stage mosquitoes.

\[
K = K_0 \frac{R(t)}{\bar{R}}
\]

Rainfall is modeled from Fourier analyses of average rainfall. This is from \cite{winskill_us_2017}

\[
R(t) = g_0 + \sum_{i=1}^3 g_i cos(2\pi t i) + h_i sin(2\pi t i)
\]

\subsubsection*{Adult female mosquitoes}

We modeled adult female mosquitoes to track the infectious population, fecundity and mosquito mortality. Mosquitoes emerge susceptible, undergo an EIP and then become infected.

The adult population can be modeled either individually or with ODEs.

\begin{gather*}
    \frac{dS^v_m}{dt} = -\Lambda_M^v S_m + \beta_{em}^v(t) - \mu^v S^v_m \\
    \frac{dE^v_m}{dt} = \Lambda_M^v S^v_m + \beta_{em}^v(t) - \Lambda^v(t - \tau) S^v_m(t - \tau) \exp(-\mu^v\tau) - \mu^v E^v_m \\
    \frac{dI^v_m}{dt} = \Lambda_M^v(t - \tau) S^v_m(t - \tau) \exp(-\mu^v\tau) - \mu^v I^v_m
\end{gather*}

We created an individual-based alternative which uses Bernoulli processes to model infection and death.

(TODO graph of adult mosquito state transitions)
% TODO image

\subsection*{Human infection}

Human infection states are stored for each individual. We use Bernoulli processes every time step to transition individuals between infection these states.

(TODO graph of human infection state transitions)
%TODO: Graph of infection state transitions

\subsubsection*{Immunity}

Humans have several layers of protection against infection. Pre-erythrocytic, clinical, detectable and severe immunities are calculated by $b_i, \phi_i, q_i, \theta_i$ respectively. This is from \cite{griffin_reducing_2010}.

\begin{gather*}
b_i(t) = b_0 \left( b_1 + \frac{1 - b_1}{1 + \frac{I_B}{I_{B0}}^{k_b}} \right)\\
\phi_i(t) = \phi_0 \left( \phi_1 + \frac{1 - \phi_1}{1 + \frac{I_{CA}(i, t) + I_{CM}(i, t)}{I_{C0}}^{k_c}} \right) \\
q_i(t, a) = d_1 \left( d_1 + \frac{1 - d_{min}}{f_D(i, a)\frac{1 + I_D(i, t)}{I_{D0}}^{k_D}} \right) \\
\theta_i(t, a) = \theta_0 \left( \theta_1 + \frac{1 - \theta_1}{1 + f_v(i, a)\frac{I_{VA}(i, t) + I_{VM}(i, t)}{I_{V0}}^{k_v}} \right)
\end{gather*}

where,

\begin{gather*}
f_v(i, a) = 1 - \frac{1 - f_{v0}}{1 + \frac{a}{a_v}^{\gamma_v}} \\
f_D(i, a) = 1 - \frac{1 - f_{D0}}{1 + \frac{a}{a_D}^{\gamma_D}}
\end{gather*}

Each human's immunity is tracked by the variables, $I_B, I_{CA}, I_{CM}, I_D, I_{VA}, I_{VM}$. These variables are incremented each time a human is exposed and then decay exponentially each time step.

\subsubsection*{Onward infectiousness}

Humans drive infection towards mosquitoes. This is from \cite{griffin_reducing_2010}

\[
\Lambda^v_M(t) = \alpha^v \sum_i c(i, t) \pi(i)
\]

The infectivity towards mosquitoes, $c$, is stored per individual. It is updated once an individual changes state.

\[
c(i, t) =
\begin{cases} 
  0  & state(i, t) \in \{S, T\} \\
  c_D & state(i, t) = D \\
  c_U & state(i, t) = U \\
  c_A(i, t) & state(i, t) = A
\end{cases}
\]

where,

\[  c_A(i, t) = c_U + (c_D - c_U) q_i \gamma^1 \]

\subsection*{Interventions}

These are some of the main interventions we model.

\subsubsection*{Vector controls}

Vector controls directly affect mosquito death rates, biting rates and force of infection towards mosquitoes.

% List death rate equation
The death rate is

\[
\mu^v = -f_R^v \log(p_1^vp_2^v)
\]

Where,

\begin{gather*}
p_1^v = \frac{p_{10}^vW^v}{1 - Z^vp^v_{10}} \\
p_{10}^v = \exp(-\mu_0^v\delta^v_{10}) \\
p_2^v = \exp(-\mu_0^v\delta^v_2) \\
f_R^v = \frac{1}{\delta^v_1 + \delta^v_2}
\end{gather*}

% List biting rate after vector controls
The biting rate is

\[
\alpha^v = Q^v f^v_R
\]

where,

\[
Q^v = 1 - \frac{1 - Q_0^v}{W^v}
\]

% List FOIM after vector controls
The effective biting rate for each human is

\[
\lambda^v_i = \frac{\alpha^v\pi_iy^v_i}{\sum_i \pi_i w^v_i}
\]

where,

\begin{gather*}
    W^v = 1 - Q_0^v + Q_0^v \sum_i \pi_iw_i^v\\
    Z^v = Q_0^v \sum_i \pi_iz_i^v
\end{gather*}

Vector controls lead to three possible outcomes. The probabilities of these outcomes are modelled as follows:

\begin{gather*}
    w_i = 1 - \Phi_I + \Phi_B(1 -  r_S)s_Ns_S + (\Phi_I - \Phi_B)(1 - r_S)s_S\\
    y_i = 1 - \Phi_I + \Phi_B(1 -  r_S)s_N + (\Phi_I - \Phi_B)(1 - r_S)\\
    z_i = \Phi_B(1 - r_S)r_N + \Phi_Ir_S
\end{gather*}

This is from \cite{griffin_reducing_2010}

If IRS is disabled, $\Phi_I = r_S = 0, s_S = 1$. If LLINs are disabled $\Phi_B = r_N = 0, s_N = 1$.

Otherwise,

\begin{gather*}
    s_S = \frac{k^\prime_S}{k_0} \\
    r_S = \left(1 - \frac{k^\prime_S}{k_0}\right)\left(\frac{j^\prime_S}{j^\prime_S + l^\prime_S}\right) \\
    d_S = \left(1 - \frac{k^\prime_S}{k_0}\right)\left(\frac{l^\prime_S}{j^\prime_S + l^\prime_S}\right)
\end{gather*}

And,

\begin{gather*}
    s_N = 1 - r_N - d_N \\ % do we want to show this??
    r_N = (r_{N0} - r_{NM})\exp(-t\gamma_N) + r_{NM} \\
    d_N = d_{N0}\exp(-t\gamma_N)
\end{gather*}

This is from \cite{sherrard-smith_systematic_2018}

\subsubsection*{Treatment}

Treatment diverts humans from clinical and severe infection.

Treatment also provides prophylaxis which is modelled with a Weibull curve.

And treatment leads to a reduction in onward infectiousness by a factor of $rel_c$

This is from \cite{okell_contrasting_2014}

\subsubsection*{Mass drug administration}

Mass drug administration can be used to induce prophylaxis in a large subset of the population.

MDA strategies can distribute a specific drug, at a list of time steps, with specified coverages, over a specified age range.

\subsubsection*{Vaccination}

RTS,S vaccination can be used to reduce the probability of infection.

We use an antibody profile for vaccine and booster efficacy:

\[
V(t) = V_{max}\left(1 - \frac{1}{1 + \frac{CS(t)}{\beta}^\alpha_{rtss}}\right)
\]

Where the antibody level is gamma distributed

\[
CS(t) = CS_{peak}\left(1 + \left(\frac{r_{peak}t}{k_{PL}}\right)^{k_{PL}}\right)
\]

after a booster dose:

\[
CS(t) = CS_{boost}\left(1 + \left(\frac{r_{boost}(t - t_{boost})}{k_{PL}}\right)^{k_{PL}}\right)
\]

RTS,S vaccines can be distributed in an EPI strategy %what is this

%How does this work

RTS,S vaccines can also be distributed in a mass campaign.

\subsubsection*{Correlation}

For interventions where a subset is selected. We correlate the sub populations between interventions with the following formulae

For every round of each intervention, individuals are included in a sub-population if

\begin{gather*}
    z_{ijt} \leq 0, \\
    z_{ijt} \sim N(u_{ij}, 1)
\end{gather*}

We find a $u_{ij}$ to correlate target populations between rounds.

\[u_{ij} \sim MVN(u_{0j}, V)\]

Where V is a matrix of I x J. and

\[
V_{jk} =
\begin{cases} 
  \sigma_j^2  & j = k \\
  \sigma_j\sigma_k\rho_{jk} & \text{otherwise}
\end{cases}
\]

The marginal standard deviations of $u_{ij}$ are given by:

\[\sigma_j = \sqrt{\frac{\rho_j}{1 - \rho_j}}\]

We find a $u_{0j}$ to create a coverage of $P_j$ with

\[ u_{0j} = -\Phi^{-1}(P_j)\sqrt{1 + \sigma_j^2} \]

\subsection*{Equilibrium}

We provide a method to initialise the model for an equilibrium EIR for non-seasonal dynamics.

\subsubsection*{Human infection states}

The human infection states ${S, D, A, U}$ are sampled based on the equilibrium proportion in the given state for the given individual's age.

The proportions are given by:

% is it ok to disregard treatment like this?
\begin{gather*}
    D_i = \frac{r_{i-1}D_{i-1} + \phi_i\Lambda_iY_i}{\beta_{D_i}} \\
    A_i = \frac{r_{i-1}A_{i-1} + (1-\phi_i)\Lambda_iY_i + r_DD_i}{\beta_{A_i} + (1-\phi_i)\Lambda_i} \\
    U_i = \frac{r_{i-1}U_{i-1} + r_{A}A_{i}}{\beta_{U_i} + (1-\phi_i)\Lambda_i} \\
    S_i = \frac{r_{i-1}A_{i-1} + (1-\phi_i)\Lambda_iY_i + r_DD_i}{\beta_{Ai} + (1-\phi_i)\Lambda_i}
\end{gather*}

where,

\begin{gather*}
    Y_i = S_i + A_i + U_i = \frac{\pi_i - b_{D_i}}{1 + a_{D_i}} \\
    a_{D_i} = \frac{\phi_i\Lambda_i}{\beta_{D_i}} \\
    \beta_{D_i} = r_D + r_i + \eta \\
    \beta_{A_i} = \phi_i\Lambda_i + r_A + r_i + \eta \\
    \beta_{U_i} = \Lambda_i + r_U + r_i + \eta
\end{gather*}

\subsubsection*{Immunity}

Each human is assigned the average immunity level based on their age.

The average immunity levels are:

\[ I_i = \frac{F_i + r_i + \eta I_{i-1}}{\frac{1}{d} +(r_i + \eta)} \]

where $F$ depends on the type of immunity:

\begin{gather*}
    F_{B_i} = \frac{\varepsilon_i}{\varepsilon_i u_B + 1} \\
    F_{C_i} = \frac{\Lambda_i}{\Lambda_i u_B + 1} \\
    F_{D_i} = \frac{\Lambda_i}{\Lambda_i u_D + 1}
\end{gather*}

\subsubsection*{Adult mosquito infectious states}

The total number of female adult mosquitoes $M$ can be calculated from the equilibrium EIR $\varepsilon^\ast$ and force of infection towards mosquitoes $\Lambda_M^v$.

\[
M_v = \frac{\varepsilon^\ast}{\sum_v p_v \alpha_v Q0_v \frac{\Lambda_M^v \exp(\mu \tau)}{\Lambda_M^v + \mu}}
\]

where the equilibrium $\Lambda_M^v$ can be estimated from the initialised human population.

\subsubsection*{Mosquito aquatic states}

The total number of mosquitoes in the aquatic states can be calculated by:

\begin{gather*}
    E = 2\omega\mu d_L(1 + d_P\mu_P)M \\
    L = 2\mu d_L(1 + d_P\mu_P)M \\
    P = 2d_P\mu M
\end{gather*}

where,

\begin{multline*}
\omega = - \frac{1}{2}\left(\gamma\frac{\mu_L}{\mu_E} - \frac{d_E}{d_L} + (\gamma - 1) \mu_L d_E\right) + \\
\sqrt{\frac{1}{4}\left(\gamma\frac{\mu_L}{\mu_E} - \frac{d_E}{d_L} + (\gamma - 1) \mu_L d_E\right)^2 + \gamma\frac{\beta\mu_L d_E}{2\mu_E\mu d_L(1 + d_P\mu_P)}}
\end{multline*}

\section*{Results}

\subsection*{Consistency with other models}

Compare with ICDMM and legacy

\subsection*{Strategy exploration}

Show some graphs of hybrid vaccine distribution combined with SMC

\subsection*{Detailed global runs}

Show some aggregated global run data

\section*{Discussion}

\subsection*{Differences with legacy model}

Point out the differences in global runs, higher incidence at high EIR.

\subsection*{Flexibility in the new model}

Hybrid vaccine strategies

\subsection*{Equilibrium}

Setting a good-enough equilibrium leads to quick convergence.

\subsection*{Seasonal slowdowns}

Highly seasonal dynamics lead to a slow ODE.

\section*{Conclusions}

This model works for global runs and for flexible strategy exploration

The code is available on github (refs)

\section*{List of abbreviations}

%%%%%%%%%%%%%%%%%%%%%%%%%%%%%%%%%%%%%%%%%%%%%%
%%                                          %%
%% Backmatter begins here                   %%
%%                                          %%
%%%%%%%%%%%%%%%%%%%%%%%%%%%%%%%%%%%%%%%%%%%%%%

\begin{backmatter}

\section*{Competing interests}
  The authors declare that they have no competing interests.

\section*{Author's contributions}
    Text for this section \ldots

\section*{Acknowledgements}
  Text for this section \ldots
%%%%%%%%%%%%%%%%%%%%%%%%%%%%%%%%%%%%%%%%%%%%%%%%%%%%%%%%%%%%%
%%                  The Bibliography                       %%
%%                                                         %%
%%  Bmc_mathpys.bst  will be used to                       %%
%%  create a .BBL file for submission.                     %%
%%  After submission of the .TEX file,                     %%
%%  you will be prompted to submit your .BBL file.         %%
%%                                                         %%
%%                                                         %%
%%  Note that the displayed Bibliography will not          %%
%%  necessarily be rendered by Latex exactly as specified  %%
%%  in the online Instructions for Authors.                %%
%%                                                         %%
%%%%%%%%%%%%%%%%%%%%%%%%%%%%%%%%%%%%%%%%%%%%%%%%%%%%%%%%%%%%%

% if your bibliography is in bibtex format, use those commands:
\bibliographystyle{bmc-mathphys} % Style BST file (bmc-mathphys, vancouver, spbasic).
\bibliography{bib}      % Bibliography file (usually '*.bib' )
% for author-year bibliography (bmc-mathphys or spbasic)
% a) write to bib file (bmc-mathphys only)
% @settings{label, options="nameyear"}
% b) uncomment next line
%\nocite{label}

% or include bibliography directly:
% \begin{thebibliography}
% \bibitem{b1}
% \end{thebibliography}

%%%%%%%%%%%%%%%%%%%%%%%%%%%%%%%%%%%
%%                               %%
%% Figures                       %%
%%                               %%
%% NB: this is for captions and  %%
%% Titles. All graphics must be  %%
%% submitted separately and NOT  %%
%% included in the Tex document  %%
%%                               %%
%%%%%%%%%%%%%%%%%%%%%%%%%%%%%%%%%%%

%%
%% Do not use \listoffigures as most will included as separate files

\section*{Figures}
  \begin{figure}[h!]
  \caption{\csentence{Sample figure title.}
      A short description of the figure content
      should go here.}
      \end{figure}

\begin{figure}[h!]
  \caption{\csentence{Sample figure title.}
      Figure legend text.}
      \end{figure}

%%%%%%%%%%%%%%%%%%%%%%%%%%%%%%%%%%%
%%                               %%
%% Tables                        %%
%%                               %%
%%%%%%%%%%%%%%%%%%%%%%%%%%%%%%%%%%%

%% Use of \listoftables is discouraged.
%%
\section*{Tables}
\begin{table}[h!]
\caption{Drug parameters}
      \begin{tabular}{l | cccc}
        \hline
        Drug acronym & efficacy & rel\_c & shape & scale \\ \hline
        DHA PQP & .95 & 0.09434 & 4.4 & 28.1\\    
        AL & .95 & 0.05094 & 11.3 & 10.6\\
        SP AQ & 0.9 & 0.32 & 4.3 & 38.1\\ \hline
      \end{tabular}
\end{table}

%%%%%%%%%%%%%%%%%%%%%%%%%%%%%%%%%%%
%%                               %%
%% Additional Files              %%
%%                               %%
%%%%%%%%%%%%%%%%%%%%%%%%%%%%%%%%%%%

\section*{Additional Files}
  \subsection*{Additional file 1 --- Sample additional file title}
    Additional file descriptions text (including details of how to
    view the file, if it is in a non-standard format or the file extension).  This might
    refer to a multi-page table or a figure.

  \subsection*{Additional file 2 --- Sample additional file title}
    Additional file descriptions text.


\end{backmatter}
\end{document}
